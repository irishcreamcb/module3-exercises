\documentclass{article}
\usepackage{amsmath}
\usepackage{amsthm,amssymb}
\usepackage[a4paper,left=25mm,right=25mm,top=30mm,bottom=30mm]{geometry}
\usepackage{fancyhdr}
\usepackage{titlesec}
\usepackage{enumerate} 
\usepackage{graphicx}
\usepackage[dvipsnames]{xcolor}
\usepackage{tikz} 
\usepackage{cancel}
\usepackage{parskip}
\usepackage[condensed,light,math]{iwona}
\usepackage[T1]{fontenc}
\usepackage{booktabs}

\title{probability exercises}
\author{emilianna louise abundo limlengco} 
\date{\today} 

\fboxsep=4pt
\RenewDocumentCommand{\footnoterule}{}{\vfill\kern-3pt \hrule width 0.4\columnwidth\kern2.6pt} %yoinked from LSE

\RenewDocumentCommand{\labelitemi}{}{$\rightarrow$}
\RenewDocumentCommand{\labelenumi}{}{\colorbox{pink}{\textbf{\arabic{enumi}}}}
\RenewDocumentCommand{\labelenumii}{}{\colorbox{CornflowerBlue!50}{\textbf{\alph{enumii}}}}

\NewDocumentEnvironment{solution}{}{%
    \RenewDocumentCommand{\qedsymbol}{}{$\blacksquare$}
    \begin{proof}[Solution]
}{\end{proof}}

\makeatletter
\newcommand*{\defeq}{\mathrel{\rlap{%
                     \raisebox{0.3ex}{$\m@th\cdot$}}%
                     \raisebox{-0.3ex}{$\m@th\cdot$}}%
                     =}
\makeatother

\begin{document} 

\section*{How to Use this Reviewer}
Hello! This is a compilation of solved exercises for module 3 of MATH 51.4. All of these exercises are taken straight from Aldrich and Cisco's course notes, so you can expect tests 
to be very similar to the items given. Some items are bound to be a little bit trickier than others, so I'll note when these items show up.\par Normal items will look like this:\begin{enumerate} 
    \item A very easy math problem. What's 1 + 1?
\end{enumerate} 
whereas difficult problems will be soulless, like this:\begin{enumerate}\setcounter{enumi}{1}
    \RenewDocumentCommand{\labelenumi}{}{\fcolorbox{magenta}{white}{\textbf{\arabic{enumi}}}}
    \item A very difficult math problem. Prove that $\displaystyle \binom{2n}{n} < 2^{2n-2},~\forall n \geq 5$ using induction. 
\end{enumerate} I might also include warnings in my \textbf{Nerd Interjections!}\par
\parindent=25pt \begin{minipage}[t]{.14\textwidth}
    \vspace{0pt}
    \includegraphics[width=2cm]{nerd_maddy.png}
\end{minipage}%
\fbox{
\begin{minipage}[t]{.76\textwidth}
    \vspace{0pt}
    \textbf{Nerd Interjection!}\footnote{Image from @Ellem\_\_ on Twitter.} These sections are for me to remind you of some necessary information to solve the problems, elaborate on 
    something that I think isn't all that clear with just pure math symbols, give a helpful theorem, be an annoying piece of shit, anything, really! Just think of it as a tips and tricks section. 
\end{minipage}%
}\parindent=0pt \par I also have another section called \textbf{Can we Prove it?} (unfortunately lacking a cute picture to go along with it), where I include some interesting, not really necessary, but 
nonetheless relevant proofs. So far, these two are my only two gimmicks, but I might add more in the future.\par
\fbox{\begin{minipage}[t]{0.98\textwidth}
    \vspace{0pt} 
    \textbf{Can we Prove it?} This is just a random proof I yoinked from our homeworks.\begin{proof} 
        ($ \implies $) Let $ x \in (A \cap B) \setminus C $. Then, $ x \in (A \cap B)$ and $ x \notin C $. \\
        \phantom{($ \implies $)} Since $x \in (A \cap B)$, $ x \in A$ and $ x \in B$. \\
        \phantom{($ \implies $)} Since $x \in A$ and $x \notin C$, $x \in (A \setminus C) $. \\
        \phantom{($ \implies $)} Since $x \in B$ and $x \notin C$, $x \in (B \setminus C) $. \\
        \phantom{($ \implies $)} Thus, $x \in (A \setminus C) \cap (B \setminus C) $. \\ 
        \\
        ($ \impliedby $) Let $ x \in (A \setminus C) \cap (B \setminus C) $. Then, $ x \in (A \setminus C) $ and $ x \in (B \setminus C) $. \\ 
        \phantom{($ \impliedby $)} Since $ x \in (A \setminus C) $, $ x \in A $ and $ x \notin C $. \\
        \phantom{($ \impliedby $)} Since $ x \in (B \setminus C) $, $ x \in B $ and $ x \notin C $. \\
        \phantom{($ \impliedby $)} Since $ x \in A $ and $ x \in B $, $ x \in (A \cap B) $. \\
        \phantom{($ \impliedby $)} Thus, $ x \in (A \cap B) \setminus C $. \\
        \\ 
        Since both sides of the conditional are true, it holds that $ (A \cap B) \setminus C = (A \setminus C) \cap (B \setminus C) $. 
    \end{proof} 
\end{minipage}%
}\par
Finally, there are blue boxes to indicate when instructions aren't obvious from the question itself, or if there are similar items that can be grouped together.\par
\parindent=25pt 
    \colorbox{CornflowerBlue!50}{
    \begin{minipage}[c]{0.9\textwidth}
        \centering
        For items \#7 to \#12, we need to reevaluate our life decisions.
    \end{minipage}
    }\parindent=0pt \par 
It's very important to note that this is a \textit{work in progress!} I am human, and I will make mistakes, and I cannot finish doing all the exercises within the span of one day. If you spot anything wrong, 
please feel free to message me; I will correct it as soon as possible.\par
As a final note, these are not replacements for the modules/paying attention in class, these are supplements for them. I won't explain all the topics here, and I'll assume that you at least have 
read the basics, so don't treat these reviewers as your only source of information. Our teachers spends a lot of time on the handouts, they're really good! (except when they're wrong) With that, though, I think 
I've covered all pertinent points. Good luck, and happy studying!

\section*{3.1.1: Linear combinations}
\textit{Some notes on notation: I'm not a big fan of the boldface lowercase letters representing matrices, particularly vectors, mostly because it's a little difficult to tell and I don't 
particularly like how it looks with the font I'm using, so I'll be using the arrow on top instead to denote vectors.}\par
\begin{center}
    \colorbox{CornflowerBlue!50}{
    \begin{minipage}[c]{0.9\textwidth}
        \centering
        Instructions for items \#1 to \#6: For each set of vectors \(V\) and given vector \(\overrightarrow{w}\), determine if \(\overrightarrow{w}\) can be expressed as a linear combination of the vectors in \(V\). 
    \end{minipage}
    }
\end{center}
\begin{enumerate}
    \item \(V = \begin{Bmatrix}
        \begin{pmatrix}
            1\\-1
        \end{pmatrix},\begin{pmatrix}
            2\\3
        \end{pmatrix}
    \end{Bmatrix}; \qquad \overrightarrow{w} = \begin{pmatrix}
        30\\30
    \end{pmatrix}\)\begin{solution}
        We need to figure out if there exist two coefficients \(k_1\) and \(k_2\) such that \(k_1\overrightarrow{v_1} + k_2\overrightarrow{v_2} = \overrightarrow{w}\).
        Expressing it in a linear equation like this, we can then solve for the coefficients using Gaussian elimination, like so\[
            \begin{pmatrix}
                1&2 \\ -1&3
            \end{pmatrix}\begin{pmatrix}
                k_1\\k_2
            \end{pmatrix} = \begin{pmatrix}
                30\\30
            \end{pmatrix} \Longleftrightarrow \begin{pmatrix}
                1&2 \\ 0&1
            \end{pmatrix}\begin{pmatrix}
                k_1\\k_2
            \end{pmatrix} = \begin{pmatrix}
                30\\12
            \end{pmatrix}.
        \] This tells us that \(k_2 =12\), and substituting this into the first row gives us \(k_1=6\). Thus, there exists a unique linear combination of vectors in \(V\) that we can express \(\overrightarrow{w}\) as. 
    \end{solution}
    \begin{minipage}{.14\textwidth}
        \vspace{0pt}
        \includegraphics[width=2cm]{nerd_maddy.png}
    \end{minipage}%
    \fbox{
    \begin{minipage}{.76\textwidth}
        \vspace{0pt}
        \textbf{Nerd Interjection!} The Gaussian elimination here is fairly straightforward since we're just dealing with two variables and two equations, so I won't bother mucking up the pages with all the steps. I might elaborate a little
        bit more later on, when we get to the sets with vectors in \(\mathbb{R}^3\), but for these first three at least, it shouldn't be too hard to see how we got from one step to the next.  
    \end{minipage}%
    }
    \item \(V = \begin{Bmatrix}
        \begin{pmatrix}
            1\\-2
        \end{pmatrix},\begin{pmatrix}
            -2\\4
        \end{pmatrix}
    \end{Bmatrix}; \qquad \overrightarrow{w} = \begin{pmatrix}
        -10\\20
    \end{pmatrix}\)\begin{solution}
        We need to figure out if there exist two coefficients \(k_1\) and \(k_2\) such that \(k_1\overrightarrow{v_1} + k_2\overrightarrow{v_2} = \overrightarrow{w}\).
        Expressing it in a linear equation like this, we can then solve for the coefficients using Gaussian elimination, like so:\[
            \begin{pmatrix}
                1&-2 \\ -2&4
            \end{pmatrix}\begin{pmatrix}
                k_1\\k_2
            \end{pmatrix} = \begin{pmatrix}
                -10\\20
            \end{pmatrix} \Longleftrightarrow \begin{pmatrix}
                1&-2 \\ 0&0
            \end{pmatrix}\begin{pmatrix}
                k_1\\k_2
            \end{pmatrix} = \begin{pmatrix}
                -10\\0
            \end{pmatrix}.
        \] This tells us that there are infinitely many solutions, and thus, infinitely many linear combinations of vectors in \(V\) that we can express \(\overrightarrow{w}\) as. If we let \(k_2\defeq r,~r\in\mathbb{R}\), then \(k_1 = 2r -10\).
    \end{solution}
    \item \(V = \begin{Bmatrix}
        \begin{pmatrix}
            1\\-2
        \end{pmatrix},\begin{pmatrix}
            -2\\4
        \end{pmatrix}
    \end{Bmatrix}; \qquad \overrightarrow{w} = \begin{pmatrix}
        1\\-3
    \end{pmatrix}\)\begin{solution}
        We need to figure out if there exist two coefficients \(k_1\) and \(k_2\) such that \(k_1\overrightarrow{v_1} + k_2\overrightarrow{v_2} = \overrightarrow{w}\).
        Expressing it in a linear equation like this, we can then solve for the coefficients using Gaussian elimination, like so:\[
            \begin{pmatrix}
                1&-2 \\ -2&4
            \end{pmatrix}\begin{pmatrix}
                k_1\\k_2
            \end{pmatrix} = \begin{pmatrix}
                1\\-3
            \end{pmatrix} \Longleftrightarrow \begin{pmatrix}
                1&-2 \\ 0&0
            \end{pmatrix}\begin{pmatrix}
                k_1\\k_2
            \end{pmatrix} = \begin{pmatrix}
                1\\-1
            \end{pmatrix}.
        \] This tells us that there is no solution, or that there is no possible linear combination of vectors in \(V\) that we can express \(\overrightarrow{w}\) as. Wow! Three for three on the three cases of solutions! That's totally not intentional. 
    \end{solution}
    \item \(V = \begin{Bmatrix}
        \begin{pmatrix}
            2\\-1\\3
        \end{pmatrix},\begin{pmatrix}
            5\\0\\4
        \end{pmatrix}
    \end{Bmatrix}; \qquad \overrightarrow{w} = \begin{pmatrix}
        -1\\-2\\2
    \end{pmatrix}\)\begin{solution}
        We need to figure out if there exist two coefficients \(k_1\) and \(k_2\) such that \(k_1\overrightarrow{v_1} + k_2\overrightarrow{v_2} = \overrightarrow{w}\).
        Expressing it in a linear equation like this, we can then solve for the coefficients using Gaussian elimination, like so:\[
            \begin{pmatrix}
                2&5\\-1&0\\3&4
            \end{pmatrix}\begin{pmatrix}
                k_1\\k_2
            \end{pmatrix} = \begin{pmatrix}
                -1\\-2\\2
            \end{pmatrix}.
        \] By inspection, though, we can see that from the second row, \(-k_1 = -2\), so \(k_1 = 2\).\footnote{LOL, I said I'd be more detailed with the Gaussian elimination but we don't even have to do it for these first two items. Love it!} Substituting this into either of the other two rows gives us \(k_2 = -1\). Thus, there exists a unique linear combination of vectors in \(V\) that 
        we can express \(\overrightarrow{w}\) as. 
    \end{solution}
    \item \(V = \begin{Bmatrix}
        \begin{pmatrix}
            2\\-1\\3
        \end{pmatrix},\begin{pmatrix}
            5\\0\\4
        \end{pmatrix}
    \end{Bmatrix}; \qquad \overrightarrow{w} = \begin{pmatrix}
        1\\1\\-1
    \end{pmatrix}\)\begin{solution}
        We need to figure out if there exist two coefficients \(k_1\) and \(k_2\) (I'm getting so bored of saying that) such that \(k_1\overrightarrow{v_1} + k_2\overrightarrow{v_2} = \overrightarrow{w}\). Then, \[
            \begin{pmatrix}
                2&5\\-1&0\\3&4
            \end{pmatrix}\begin{pmatrix}
                k_1\\k_2
            \end{pmatrix} = \begin{pmatrix}
                1\\1\\-1
            \end{pmatrix}.
        \] By inspection, we can see that from the second row, \(-k_1 = 1\), so \(k_1 = -1\). However, substituting this into the other two rows gives us different answers, namely \(k_2 = 0.6\) and \(k_2 = 0.5\).\ \(k_2\) can't have two values, 
        so this is a contradiction, implying that there is no solution, and that \(\overrightarrow{w}\) cannot be expressed as a linear combination of vectors in \(V\). 
    \end{solution}
    \item \(V = \begin{Bmatrix}
        \begin{pmatrix}
            2\\-1\\3
        \end{pmatrix},\begin{pmatrix}
            5\\0\\4
        \end{pmatrix},\begin{pmatrix}
            3\\1\\1
        \end{pmatrix}
    \end{Bmatrix}; \qquad \overrightarrow{w} = \begin{pmatrix}
        1\\-8\\12
    \end{pmatrix}\)\begin{solution}
        We need to figure out of there exist \textit{three} (YIPPIE!) coefficients \(k_1\), \(k_2\), and \(k_3\) such that \(k_1\overrightarrow{v_1} + k_2\overrightarrow{v_2} + k_3\overrightarrow{v_3} = \overrightarrow{w}\). Expressing it in a linear
        equation like this, we can then solve for the coefficients using Gaussian elimination, like so (let's actually go through the Gaussian elimination for this one):\begin{align*} 
            \begin{pmatrix}
                2&5&3\\-1&0&1\\3&4&1
            \end{pmatrix} \begin{pmatrix}
                k_1\\k_2\\k_3
            \end{pmatrix} &= \begin{pmatrix}
                1\\-8\\12
            \end{pmatrix} &\begin{matrix}\\2R_2 + R_1 \implies R_1\\3R_2 + R_3 \implies R_3 \end{matrix}&&\begin{pmatrix}
                0&5&5\\-1&0&1\\0&4&4
            \end{pmatrix} \begin{pmatrix}
                k_1\\k_2\\k_3
            \end{pmatrix} &= \begin{pmatrix}
                -15\\-8\\-12
            \end{pmatrix}\\
            &&  \begin{matrix}\\\frac{1}{5}R_1 \implies R_1\\\frac{1}{4}R_3 \implies R_3 \end{matrix}&&\begin{pmatrix}
                0&1&1\\-1&0&1\\0&1&1
            \end{pmatrix} \begin{pmatrix}
                k_1\\k_2\\k_3
            \end{pmatrix} &= \begin{pmatrix}
                -3\\-8\\-3
            \end{pmatrix}.
        \end{align*} We see here that there are two identical rows, which should already clue us in to the fact that there are infinitely many solutions. Thus, if we choose \(k_3 \defeq r,~r\in\mathbb{R}\), then \(k_2=-r-3\) and \(k_1=r+8\).
    \end{solution}
\end{enumerate}
\begin{center}
    \colorbox{CornflowerBlue!50}{
    \begin{minipage}[c]{0.9\textwidth}
        \centering
        Instructions for items \#7 to \#10: Each of the following sets of vectors \(\lbrace\overrightarrow{v_1},\overrightarrow{v_2}\rbrace\) below is special in that any vector \((x,y) \in \mathbb{R}^2 \) can be uniquely 
        expressed as a linear combination of \(\overrightarrow{v_1}\) and \(\overrightarrow{v_2}\). Find the \textit{coefficients} \(k_1\) and \(k_2\) in terms of \(x\) and \(y\) such that\[
            k_1\overrightarrow{v_1} + k_2\overrightarrow{v_2} = \begin{pmatrix}
                x\\y
            \end{pmatrix}.
        \]
    \end{minipage}
    }\end{center}
\begin{enumerate}
    \setcounter{enumi}{6}
    \item \(\overrightarrow{v_1} = \begin{pmatrix}
        4\\3
    \end{pmatrix},\qquad\overrightarrow{v_2} = \begin{pmatrix}
        1\\1
    \end{pmatrix}\)\begin{solution}
        Let \((x,y)\) be any arbitrary vector in \(\mathbb{R}^2\). Setting up the linear equation gives us\[
            \begin{pmatrix}
                4&1\\3&1
            \end{pmatrix} \begin{pmatrix}
                k_1\\k_2
            \end{pmatrix} = \begin{pmatrix}
                x\\y
            \end{pmatrix} \qquad R_1 - R_2 \implies R_2 \qquad \begin{pmatrix}
                4&1 \\ 1&0 
            \end{pmatrix} \begin{pmatrix}
                k_1\\k_2
            \end{pmatrix} = \begin{pmatrix}
                x \\x-y
            \end{pmatrix},
        \] so \(k_1 = x-y\). Substituting this into the first row gives us \(k_2 = 4y - 3x\). 
    \end{solution}
    \item \(\overrightarrow{v_1} = \begin{pmatrix}
        7\\2
    \end{pmatrix},\qquad\overrightarrow{v_2} = \begin{pmatrix}
        3\\1
    \end{pmatrix}\)\begin{solution}
        Let \((x,y)\) be any arbitrary vector in \(\mathbb{R}^2\). Setting up the linear equation gives us\[
            \begin{pmatrix}
                7&3\\2&1
            \end{pmatrix} \begin{pmatrix}
                k_1\\k_2
            \end{pmatrix} = \begin{pmatrix}
                x\\y
            \end{pmatrix} \qquad R_1 - 3R_2 \implies R_2 \qquad \begin{pmatrix}
                7&3 \\ 1&0 
            \end{pmatrix} \begin{pmatrix}
                k_1\\k_2
            \end{pmatrix} = \begin{pmatrix}
                x \\x-3y
            \end{pmatrix},
        \] so \(k_1 = x-3y\). Substituting this into the first row gives us \(k_2 = 7y - 2x\).
    \end{solution}
    \item \(\overrightarrow{v_1} = \begin{pmatrix}
        3\\-1
    \end{pmatrix},\qquad\overrightarrow{v_2} = \begin{pmatrix}
        5\\2
    \end{pmatrix}\)\begin{solution}
        Let \((x,y)\) be any arbitrary vector in \(\mathbb{R}^2\). Setting up the linear equation gives us\[
            \begin{pmatrix}
                3&5\\-1&2
            \end{pmatrix} \begin{pmatrix}
                k_1\\k_2
            \end{pmatrix} = \begin{pmatrix}
                x\\y
            \end{pmatrix} \qquad R_1 + 3R_2 \implies R_1 \qquad \begin{pmatrix}
                0&11 \\ -1&2 
            \end{pmatrix} \begin{pmatrix}
                k_1\\k_2
            \end{pmatrix} = \begin{pmatrix}
                x+3y \\y
            \end{pmatrix},
        \] so \(k_2 = \dfrac{1}{11}(x+3y)\). Substituting this into the first row gives us \(k_1 = \dfrac{1}{11}(2x-5y)\).
    \end{solution}
    \item \(\overrightarrow{v_1} = \begin{pmatrix}
        5\\-1
    \end{pmatrix},\qquad\overrightarrow{v_2} = \begin{pmatrix}
        -2\\3
    \end{pmatrix}\)\begin{solution}
        Let \((x,y)\) be any arbitrary vector in \(\mathbb{R}^2\). Setting up the linear equation gives us\[
            \begin{pmatrix}
                5&-2\\-1&3
            \end{pmatrix} \begin{pmatrix}
                k_1\\k_2
            \end{pmatrix} = \begin{pmatrix}
                x\\y
            \end{pmatrix} \qquad 5R_2 + R_1 \implies R_2 \qquad \begin{pmatrix}
                5&-2 \\ 0&13 
            \end{pmatrix} \begin{pmatrix}
                k_1\\k_2
            \end{pmatrix} = \begin{pmatrix}
                x \\x+5y
            \end{pmatrix},
        \] so \(k_2 = \dfrac{1}{13}(x+5y)\). Substituting this into the first row gives us \(k_1 = \dfrac{1}{13}(3x + 2y)\).\footnote{Yes, these answers with fractions of \textit{prime numbers larger than 10} are correct, I double-checked.}
    \end{solution}
\end{enumerate}
\pagebreak 

\section*{3.1.2: Norm and dot product}
\textit{We learned about this stuff in calculus so it shouldn't be that new, but to be honest I've forgotten everything since vectors were an afterthought in 30.24, so here's a refresher, I guess.}
\pagebreak

\section*{3.1.3: Standard matrices of linear transformations}
\textit{Oh no! Drawings! Fuck! Even if it's kind of a pain, I'd much rather do all the graphs in \LaTeX{} than use some external graphing software and take screenshots. I'm pretty familiar with tikz anyway (waw). }
\pagebreak 

\end{document}